INTRODUCTION

Language comprehension involves not only interpreting the literal meanings of words in utterances, but also understanding the communicative intentions behind what is said. Imagine that Bob bought two kinds of cookies to eat, chocolate-chip and raisin, and left them on a table. Alice approaches and says to Bob, ?I ate some of your cookies, the chocolate-chip ones.? Later, Bob is surprised to find that all of the cookies are gone. It is technically true that Alice ate some (i.e. one or more) cookies, and chocolate-chip (alongside raisin) cookies. Then why would Bob be surprised that Alice ate all the cookies?

According to Grice (1975), language users tend to follow the Cooperative Principle: to make their contribution as required, when it is required, by the conversation in which they are engaged. When people fail to meet this expectation, the conversation may fall apart and misunderstandings arise. In the previous scenario, Alice was not as informative as she could be by saying she ate all the cookies, and thus led to Bob?s misunderstanding that Alice ate some of the cookies but not all. Alice thus failed to warn against the assumption that she was making pragmatic implicatures, where speaker?s implied meaning goes beyond the conventional meaning of the utterance.

Pragmatic implicatures are common and efficient ways of implicitly expressing speakers' intended meanings; thus, implicature computation is important for proficient language use, both for adults and for children. For example, if a child asks a parent whether they could have cookies for snack, and a parent says: ?you can have some of the cookies,? the child should be able to infer that she should eat some but not all of the cookies. Thus, implicatures make up an important case study for adults? and children?s pragmatic understanding and show what they can infer about speaker?s implicit communicative intentions. Whereas adults have been shown to robustly make implicature inferences, studies have shown mixed results about children?s ability to compute implicatures, and it is not yet clear what cause children?s successes and failures in implicature computation. In the next section, we review previous research on adults? and children?s processing of implicatures.

IMPLICATURES AND PRAGMATIC UNDERSTANDING IN CHILDREN

There are two main kinds of implicatures that previous studies have examined to look at adults? and children?s processing got implicatures: scalar implicatures and ad-hoc (contextual) implicatures.^Note: these implicatures have also been identified as generalized versus particularized implicatures, but this distinction has been controversial; for the purpose of the current study, we remain agnostic about the theoretical distinction.
In the previous example of Alice?s utterance, ?I ate some of the cookies? is a scalar implicature (SI): it implicates that she ate some but not all, which relies on generating the relevant lexical scale <some, all> upon hearing the weaker term ?some? (Horn, 1972) and negating the stronger alternative (?all?). On the other hand, ?I ate chocolate-chip cookies? is an example of ad-hoc implicature, in which use of the contextually weaker term (?chocolate-chip cookies?) negates the stronger description (?chocolate-chip and raisin cookies). 

Many studies have looked at adults? and children?s processing of SI?s. It has been shown that adults readily compute scalar implicatures, even though implicature inferences are generally slower than processing of unambiguous meanings (Bott & Noveck, 2004; Grodner, Klein, Carbary, & Tanenhaus, 2010; Huang and Snedeker, 2009a). However, children often struggle on SI tasks; For example, Papafragou and Musolino (2003) presented adults and 5-year-old children with a context in which three out of three horses jumped over a fence. Then a puppet offered a description of the context by saying: ?some of the horses jumped over the fence.?? When asked whether what the puppet said was a good description of what happened, most of the adults denied it, whereas almost 90\% of the children accepted it as an adequate description. Besides this study, other numerous studies have also reported similar findings that children seem to have difficulty computing SI's (e.g. Barner, Chow, & Yang, 2009; Huang & Snedeker, 2009; Noveck, 2001; Papafragou and Musolino, 2003).

Children?s struggle on previous implicature tasks is a mystery given their early emerging sensitivity to informativeness of utterances, a pragmatic skill that is important for implicature computation. For example, to know that the sentence ?some of the horses jumped over the fence? is a bad description of a situation in which all of the horses jumped over the fence, the listener needs to be aware that saying ?some? when ?all? is true is not maximally informative. What makes children's failures on SI computation surprising is that from early age, children are adept at sensing informativeness of others' and their own utterances: children adjust informativeness of their own expressions depending on the listeners' knowledge at 3 years of age (Matthews et al., 2006); by 4 years, they provide more information when the distractor is more similar to the target and thus disambiguation is more difficult (Matthews et al., 2012); and by 5 years, they reward speakers based on the informativeness of their utterances (Katsos et al., 2011). Even at two years of age, children are able to assess informativeness of their own gestures (O?Neill, 2001) Thus, even young children seem to excel at assessing the informativeness of both their own and other people's utterances and communicative gestures, which suggests that lack of general pragmatic understanding is not the cause of their failures on SI computation. 

Then what can be potential causes of children?s failures in implicature tasks? In the next section, we consider potential contributing factors to children?s performances in previous implicature tasks. 

POTENTIAL CAUSE OF CHILDREN'S DIFFICULTY WITH SI: ACCESS TO ALTERNATIVES

Theorists have looked for other causes besides general pragmatic understanding that might contribute to children's failures in SI computation. One of these potential factors that lead to children's difficulty with SI computation is whether children have access to lexical alternatives to the term offered. Implicature computation involves generating and negating alternatives to the given term. For example, upon hearing ?some,? the listener needs to generate a stronger alternative (?all? in this case) based on the lexical knowledge, and negate that alternative. One possibility is that adults are good at generating these alternatives, whereas children are less so.

Barnet, Brooks and Bale (2011) were first to test this ?alternative hypothesis?: even though children know that there are alternatives to be negated, they may not be able to generate those relevant scalar alternatives on their own and need to rely on the context to find what the alternatives are. To test this, they presented 5-year-old children with a context in which three out of three animals, a dog, cat, and cow, were sleeping. One group of children were asked, ?are some of the animals sleeping?? and as expected from failures on previous SI tasks, most children said yes, not computing the implicature ?some but not all.? Another group of children were asked: ?are only the cat and the cow sleeping?? Children successfully rejected this statement, demonstrating their understanding of the word ?only? within the question, that asked whether the cat and the cow are sleeping whereas the dog is not. Importantly, another group of children were asked: ``are only some of the animals sleeping?'' Because children knew that the word ?only? required the alternatives to the term offered to be negated, if children did have access to alternative to ?some? (i.e., ?all?), then children would say ?no.? However, most children still said ?yes,? providing support for Barner et al.?s hypothesis.  

There have, however, been other studies on SI computation that also made access to lexical alternatives available, but failed to induce children's SI computation. Huang and Snedeker (2009b) looked at 5-year-old children's real-time processing of utterances given contexts in which there were two girls and two boys; for example, girl A had two socks, boy B also had two socks, girl C had three soccer balls and boy D had none of the items. Based on this context, girl A had two out of four, hence ?some but not all? of the socks, whereas girl C had three out of three, or ?some and indeed all? of the soccer balls. Hence, Huang and Snedeker predicted that, if children computed the implicature ?some but not all,? then children would look immediately toward the girl with socks rather than the girl with soccer balls upon hearing ?some.? However, children did not look to the correct target above chance until after the disambiguation phase (i.e., between ?socks? vs. ?soccer balls?). Additionally, Hurewitz et al. (2006) showed that when 3-4-year-olds were asked to select a picture that is the best match with the speaker's utterance with ?some but not all? implicature (e.g., ?The alligator took some of the cookies?), they failed to choose the picture that depicted the alligator taking a subset (e.g., two out of four) of the cookies. 

If the alternatives hypothesis is correct, it is rather puzzling that children's difficulty with SI computation was not resolved despite the availability of lexical alternatives in the context. However, several aspects other than alternative availability potentially imposed difficulty on children's SI computation in the designs of the previously mentioned studies. A common factor in both Huang and Snedeker?s and Hurewitz et al.?s paradigms that potentially made them difficult is relatively little salience of contrasts of alternatives for implicature computation. In Huang and Snedeker?s task, the availability of alternative relevant to SI computation was contingent upon whether children were able to group different objects or people holding those objects as constituting different sets. For example, children needed to group girl A and boy B together since they were both holding socks, then recognize that girl A is holding a partitive set of those objects relative to the entire set; then children needed to recognize that girl C was holding a complete set of soccer balls. This could have been a major cognitive load for children, especially in such short time period they were given to process these meanings before the disambiguation point. Hurewitz et al.?s paradigm had a different issue regarding the salience of contrasts: children were presented with one target picture (i.e., `took some but not all') and three distractor pictures (i.e., `took none, `took all,' `none to be taken'), rather than one (i.e., `took all'). The presence of multiple distractors could have made the contrast less salient and more difficult to pay attention to, which potentially caused children to be unaware of the scales of alternatives set up in the context. 

PROCESSING OF AD-HOC IMPLICATURES

Leaving aside the considerations for what makes specifically SI computations difficult, perhaps it would be useful to turn to a more general question: are young children (below 5 years) able to compute pragmatic implicatures at all, if we minimize processing constraints present in the implicature tasks? A useful simple test for this question is to look at children's processing of ad-hoc implicatures, which involve the knowledge of contextually-derived scales or alternatives, rather than lexically-derived alternatives. 

Papafragou and Tantalou (2004) conducted the first study to look at 4- to 6-year-old children's processing of ad-hoc implicatures. In their study, children watched as a puppet was instructed to e.g., wrap two gifts (a parrot and a doll), which led to implicit constitution of scales (i.e., the parrot, the doll, the parrot and the doll). The puppet went into a dollhouse to wrap the gifts such that children were unable to see him. When he reappeared, the experimenter asked: ?Did you wrap the gifts,? to which the puppet answered, ?I wrapped the parrot.? When asked whether the puppet should be rewarded for completing his request, children rejected to reward the puppet 90\% of the time. Based on these results, Papafragou and Tantalou claimed that children were able to compute implicatures and interpret the puppet's responses as indicating he only completed a partial set of the entire task (but see Sullivan, Davidson, & Barner, 2011). Similarly, Katsos and Bishop (2011) found that 5-year-olds give lower reward for under-informative utterances with ad-hoc scales at higher rates than for those with lexical scales (e.g. <some-all>).

Stiller, Goodman and Frank (2015) did another test of children's ability to compute ad-hoc implicatures, using a referent-selection task rather than felicity-judgment task. They presented 2- to 4-year-old children with pictures of three different faces, for instance: one wearing glasses and a top hat, another one wearing only glasses, and another wearing none of the two. Then a puppet said, ?my friend has glasses,? then asked children to pick out the correct referent. Given the context, only using the term ?glasses? implicates the referent has ?glasses but not a top-hat,? suggesting that the face with only glasses is the referent. Indeed, Stiller et al. found that children as young as 3.5 years of age chose the implicature-consistent, one-feature referent above chance when they heard utterances such as ?my friend has glasses.? This suggested that children as young as 4 years are in fact able to compute implicatures, given the availability of inferential alternatives in the context. 

YOUNGER CHILDREN'S STRUGGLE WITH IMPLICATURES

So far, we have reviewed findings that children mostly struggle with SI?s but their struggle may be due to their lack of access to alternatives and cognitive task demands. Supporting this possibility, children are shown to robustly compute ad-hoc implicatures, with first evidence at 3.5 years of age (Stiller et al., 2015). However, children below 3 years still seem to struggle with implicature computation task: even in Stiller et al., 2015?s simplified paradigm, 2- to 3-year-old children failed to choose implicature-consistent targets above chance. As mentioned above, Hurewitz et al. (2006) similarly found that 3-year-old children did not succeed on referent-selection task involving SI computation. What causes younger children's failures on implicature tasks? One possibility is that younger children in fact have poorer understanding of pragmatic implicatures; another possibility that there is an additional factor involved in these tasks, at which older children are good at and younger children are bad at, which obscures younger children's pragmatic understanding. 

One candidate for younger children's difficulty is their lack of inhibitory control to restrain choosing, or directing their attention to, more salient referents. For example, most of the experimental tasks on implicature computation required children to reject alternatives that are stronger than the term offered to them (e.g., reject ?all? upon hearing ?some?; rejecting ?glasses and top-hat? upon hearing ?glasses? given the appropriate context). These alternatives to be negated are often more perceptually and conceptually salient, and may be harder to draw attention away from.  This inhibitory hypothesis is in line with constant effects of salience in younger children?s word learning (Yurovsky and Frank, 2015) and findings that younger children have difficulty with executive control, and the ability for inhibitory control continues to develop throughout childhood (Davidson et al., 2006; Gerardi, 2000). Indeed, children in Huang and Snedeker?s study looked more at the relatively more salient potential referent (e.g., girl with three soccer balls) compared to less salient one (e.g, girl with two socks) before they heard disambiguation to know that the referent was the less salient object. 

This hypothesis gives rise to interesting predictions for tradeoffs between pragmatic computation and inhibitory demand. On one hand, salience contrast between inferential target and distractor predicts that, as contrast increases (i.e. distractor becomes much more salient than the target, with more items: face with glasses (target) vs. face with glasses, top-hat, and bow tie), it will be more difficult for children to identify the inferential target that is less salient. On the other hand, however, computational model of pragmatic inferences (Frank & Goodman, 2012) suggests that the increased contrast strengthens the implicature in question: ?My friend has glasses? is more likely to refer to the face with glasses as opposed to face with two other items than one item.  

THE CURRENT STUDY

The review of previous studies on children?s implicature processing above helps identify two important factors to be accounted for in next efforts in queue: a simple paradigm to reduce extraneous cognitive load for children in computing implicatures (e.g. visual access to contextual alternatives in an easy-to-see grouping; as in Stiller et al., 2015) But we also need methodological tools that allow us to examine more closely children?s decision-making process (as in Huang and Snedeker, 2009b), to be able to identify what contributes to children?s successes and struggles in implicature processing. 

We present a set of studies that take a closer look at children?s implicature processing abilities. The goals of the current studies were (1) to confirm that children robustly compute pragmatic implicatures starting at an early age (3-4 years); and (2) to identify potential contributing factors to children?s inferential processes. To do this, we implement two different methodological tools to look at timecourse and speed of implicature processing: eye-tracking and tablet paradigms. Eye-tracking is a useful tool for looking at real-time inferential decision-making process, and may help identify factors that contribute to how much fast decisions are made, and how much attention is allocated to each potential referent, depending on utterances and contexts. Tablets provide an engaging way to collect data from young children, and is useful in that it yields comparable data to that of behavioral paradigms, while making it possible to examine the accuracy and speed (reaction time) of children?s judgments (Frank, Sugarman, Horowitz, Lewis, & Yurovsky, 2016). These two tools can complement each other in their strengths as naturalistic and engaging methods that reveal time-sensitive information about children?s implicature processing.

In Experiment 1, we used an eye-tracking paradigm with a simplified design to test children's ability to compute ad-hoc implicatures: children saw two items with single or double features (e.g., a plate with a carrot and banana, and a plate with only a carrot) and heard the common feature (e.g., carrot) named, suggesting an implicature (i.e., a carrot but not a banana). We identify one potential factor that contributes to children?s implicature processing: salience-pragmatic tradeoff, or tradeoff between perceptual salience versus strength of implicature. In Experiment 2 and 3, we explore the effect of salience-pragmatic tradeoff, using eye-tracking (Experiment 2) and tablet paradigm (Experiment 3). We find that children as young as 3 years of age robustly compute ad-hoc implicatures in our paradigm, and we report some evidence for salience affecting young children (2-year-olds)?s implicature computation performance.

